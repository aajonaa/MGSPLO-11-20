\documentclass[11pt]{article}
\usepackage[a4paper,margin=1in]{geometry}
\usepackage{amsmath,amssymb}
\usepackage[numbers,sort&compress]{natbib}
\usepackage{hyperref}

\title{Wrapper Feature Selection for Pneumonia Diagnosis from Routine Blood Markers using MGSPLO}
\author{}
\date{}

\begin{document}
\maketitle

\section{Introduction}

Pneumonia is a common and serious infection worldwide. It causes substantial illness and death and places a heavy burden on health systems \citep{gbd2019_lri_2022}. Community-acquired pneumonia (CAP) ranges from mild outpatient disease to severe illness with respiratory failure and sepsis that requires intensive care \citep{musher_jama_2014,torres_lancet_cap_2021}.

In routine practice, diagnosis combines symptoms and signs with chest imaging and basic laboratory tests \citep{franquet_imaging_cap_2018,torres_lancet_cap_2021}. Chest radiographs may be non-diagnostic early in the course of disease \citep{okada_respiration_2018}. In such cases a CT scan can reveal pneumonia even when the concurrent radiograph is negative \citep{self_ct_only_pneumonia_2018,claessens_ct_2015}. These gaps motivate data-driven tools that work well with information available at the bedside.

Host-response blood biomarkers are widely used in suspected pneumonia. Typical examples are complete blood count (CBC) indices and acute-phase reactants such as C-reactive protein (CRP) and procalcitonin (PCT) \citep{diagnostics_pct_crp_2023,amjmed_crp_2008}. Higher leukocyte counts with neutrophil predominance and elevated CRP have been linked to disease severity and outcomes \citep{amjmed_crp_2008,jpm_crp_cap_2022}. PCT-guided strategies can also support antibiotic stewardship in acute respiratory infections, including CAP \citep{schuetz_cid_2012,schuetz_cochrane_2017}.

Ratios derived from the CBC summarize inflammatory imbalance. The neutrophil-to-lymphocyte ratio (NLR) has shown prognostic value in hospitalized CAP \citep{plosone_nlr_serial_2021,biomedicines_nlr_2024}. Combining NLR with clinical scores can further improve risk stratification in older adults \citep{ijgm_nlr_curb65_2021,bmcgeriatrics_nlr_2025}. Related ratios such as the platelet-to-lymphocyte ratio (PLR) and monocyte-to-lymphocyte ratio (MLR) also correlate with severity indices and adverse outcomes \citep{jcm_2024_cap_biomarkers,infection_2023_nlr_plr}.

At the same time, machine-learning (ML) models that integrate multiple clinical and laboratory variables show strong discrimination for diagnosis and prognosis in CAP \citep{cbm_2024_cap_clinlab,fbioe_2022_cap_ml}. Some recent studies report gains for outcomes such as mortality and length of stay using only routinely collected features \citep{srep_2024_lowcode_cap,bmjopenresresp_2025_severe}. By contrast, a large ML literature still emphasizes imaging-derived features from chest radiography or CT \citep{kermany_cell_2018,khan_jimaging_2024}.

From a data perspective, even a modest blood panel becomes high dimensional once derived ratios and simple interactions are included. Training directly on such expanded sets increases overfitting risk and reduces interpretability, especially with limited sample sizes. Feature selection (FS) aims to find a compact, informative subset that preserves or improves performance and remains clinically interpretable \citep{guyon_elisseeff_2003,chandrashekar_sahin_2014}. FS methods are commonly grouped as filters, embedded methods, and wrappers \citep{battiti_mi_1994,tibshirani_lasso_1996}. Wrappers evaluate candidate subsets using a downstream classifier and naturally capture interactions, though at higher computational cost \citep{kohavi_wrappers_1997}. The search space is combinatorial: with $D$ candidates there are $2^{D}-1$ nonempty subsets \citep{blum_langley_1997}.

Because exhaustive wrapper search is infeasible, metaheuristic optimization is popular in FS. These algorithms are derivative-free and designed to escape local optima in high-dimensional landscapes \citep{xue_survey_ec_fs_2016}. Classic examples include Particle Swarm Optimization (PSO) and Differential Evolution (DE) \citep{kennedy_pso_1995,storn_de_1997}. Newer nature-inspired methods—such as Grey Wolf Optimizer (GWO) and Harris Hawks Optimization (HHO)—are also widely used \citep{mirjalili_gwo_2014,heidari_hho_2019}. The Slime Mould Algorithm is another recent approach \citep{li_sma_2020}. Despite strong empirical performance, metaheuristics can exhibit premature convergence, an unstable exploration–exploitation balance, and hyperparameter sensitivity \citep{boussaid_metaheuristics_2013}. No-free-lunch results further underline that careful design and tuning are problem dependent \citep{nfl_1997}.

Many enhanced or hybrid strategies have been proposed to improve convergence and solution quality. Examples include comprehensive-learning/topology variants of PSO and opposition-based learning \citep{clpso_2006,tizhoosh_obl_2005}. Local search and differential-evolution hybrids can also help maintain diversity and refine elites \citep{rahnamayan_ode_tevc_2008,blum_roli_csur_2003}. For binary FS, transfer functions provide a principled way to convert continuous updates into probabilistic inclusion decisions \citep{mirjalili_vshape_2013}.

\paragraph{Our approach.}
We propose a \textbf{Multi-Guide Slime-Predator Learning Optimizer (MGSPLO)} for wrapper FS over routine blood and inflammatory markers. MGSPLO is designed to (i) use \emph{multi-anchor guidance} so each candidate learns from several strong subsets, (ii) preserve population diversity via adaptive exploration inspired by predator–prey dynamics and slime-mould-like oscillations, and (iii) apply a tailored \emph{binary mapping} for fine-grained, probability-driven feature inclusion. In our pipeline, MGSPLO wraps a Support Vector Machine (SVM) and uses cross-validated performance as the fitness signal \citep{cortes_vapnik_1995,kohavi_cv_1995}.

\paragraph{Study setting and goals.}
We analyze adults evaluated for pneumonia, with age, comorbidities, CBC (including differentials and red cell/platelet indices), and CRP measured at presentation. From these we derive composite inflammation ratios (e.g., NLR, PLR) to build a feature pool that reflects clinical practice and biomarker evidence. Our goals are threefold. First, we quantify the diagnostic value of routine blood and inflammatory markers—alone and in combination—for pneumonia detection. Second, we test whether MGSPLO-driven wrapper FS can identify small, stable, interpretable subsets that match or improve SVM performance versus using all markers or standard FS baselines. Third, we relate the selected markers to current biomarker literature.

\paragraph{Contributions.}
\begin{itemize}
  \item We introduce \textbf{MGSPLO}, a multi-guide, diversity-preserving metaheuristic tailored to wrapper FS in high-dimensional clinical feature spaces.
  \item We provide a \textbf{binary} MGSPLO variant with a refined transfer mechanism enabling granular, probability-driven inclusion of blood routine and inflammatory markers.
  \item We build an end-to-end \textbf{SVM} CAP diagnosis pipeline that integrates MGSPLO with cross-validated evaluation and compare against standard FS techniques and alternative wrapper metaheuristics.
  \item On a real-world cohort, \textbf{MGSPLO+SVM} achieves competitive or better accuracy while selecting compact subsets of routine markers, supporting use where imaging resources are limited.
\end{itemize}

\bibliographystyle{unsrtnat}
\bibliography{refs}
\end{document}
